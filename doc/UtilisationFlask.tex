\documentclass[12pt,a4paper]{article}
\usepackage[utf8x]{inputenc}
\usepackage{dirtree}
\usepackage{xcolor}
\usepackage{minted}
\usemintedstyle{monokai}
\definecolor{bg}{HTML}{282828}
\definecolor{bghtml}{HTML}{ADADAD}
\definecolor{Text}{HTML}{F8F8F2}
\AtBeginEnvironment{minted}{\color{Text}}

\begin{document}


\section{Architecture du projet}

\dirtree{%
.1 fifo\_and\_lifo.
   .2 app/.\DTcomment{Dossier contenant les modules tiers}.
   .2 doc/.
        .3 UtilisationFlask \DTcomment{Ce document}.
   .2 static/.
        .3 assets/ \DTcomment{Dossier contenant les fichiers tiers}.
        .3 css/ \DTcomment{Dossier des fichiers CSS du projet}.
        .3 db/ \DTcomment{Dossier contenant la(es) base(s) de données}.
        .3 images/. 
        .3 pdfs/.
        .3 videos/.
   .2 template/\DTcomment{Dossier contenant les fichiers HTML du projet}.
   .2 main.py.\DTcomment{Programme principal}.
   .2 README.md.
   .2 fifoandlifo.ini\DTcomment{Fichier de configuration uWSGI}.
}

\newpage

\section{Utilisation de Flask}

\subsection{Code minimal fonctionnel d'une application}
 %  A placer en entête


\begin{minted}[bgcolor=bg]{python}

from flask import Flask
app = Flask(__name__)

@app.route('/')
def hello_world():
    return 'Hello, World!'

\end{minted}



\subsection{Créer et afficher une page HTML}

\subsubsection{Code HTML}
  Les fichiers statiques doivent être déposé dans le dossier static

Ainsi pour créer une page \texttt{bonjour.html}, qui contient l'image \texttt{bonjour.html} placée dans \texttt{static/images}

 %  A placer en entête

 %\usepackage{minted}
 %\usemintedstyle{monokai}
 %\definecolor{bg}{HTML}{282828}
 %\definecolor{Text}{HTML}{F8F8F2}
 %\AtBeginEnvironment{minted}{\color{Text}}


\begin{minted}[bgcolor=bghtml]{HTML}
<html>
  <head>
    <meta charset="utf-8">
    <title>Bonjour</title> 
  </head>
  <body style="background-image: url('/static/images/bonjour.jpg');background-size: cover;">
    
<h1>Bonjour l'équipe</h1>
  </body>
</html>
\end{minted}


 %  A placer en entête

 %\usepackage{minted}
 %\usemintedstyle{monokai}
 %\definecolor{bg}{HTML}{282828}
 %\definecolor{Text}{HTML}{F8F8F2}
 %\AtBeginEnvironment{minted}{\color{Text}}

\subsubsection{Attribuer une fonction Python à une URL}







Maintenant, il faut que le navigateur puisse avoir accès à cette page. 

\par 
\medskip
On lie donc cette page HTML à une fonction python puis on lie la fonction à l'URL souhaitée. \par 
\texttt{return render\_template('/template\_de\_la\_page')}

\par 
\medskip


Pour lier la fonction à l'URL, on utilise un décorateur. \par 
\texttt{app.rout('/url')}

\begin{minted}[bgcolor=bg]{python}
 @app.route('/bonjour')
 def bonjour():
  return render_template('/bonjour.html')
\end{minted}

\begin{minted}[bgcolor=bg]{python}
@app.route('/bonjour')
def fct_bonjour():
  return render_template('bonjour.html') 

\end{minted}


\subsection{Communication Python ->  HTML}
\begin{enumerate}
  \item On peut utiliser des variables Python dans le code en les encadrant par des doubles accolades $\{\{  ma variable  \}\}$.



  \item On peut utiliser des boucles et des instrtuctions conditionnelles en les encadrant par $\lbrace\%  mon instruction  \%\rbrace$.
\end{enumerate}

\begin{multicols}2

Dans le fichier \texttt{main.py}

\begin{minted}[bgcolor=bg]{python}
@app.route('/bonjour')
def bonjour():
    equipe = ['denis', 'alan', 'joris', 'eric', 'etienne']
    return render_template('/bonjour.html',
      equipe=equipe)
\end{minted}

\columnbreak


Dans le fichier bonjour.html

\begin{minted}[bgcolor=bghtml]{html}
<html>
  <head>
    <meta charset="utf-8">
    <title>Bonjour</title> 
  </head>
  <body style="background-image: url('/static/images/bonjour.jpg');
  background-size: cover;">
    
<h1>Bonjour l'équipe</h1>

<ul>

  <li>{{nom}}</li>

</ul>

  </body>
</html>
\end{minted}

\end{multicols}


 \section{Communication HTML -> Python}
 

Un des moyens de communication d'une page HTML vers  le serveur est de passer par des formulaires.

On utilisera la methode \texttt{POST} pour faire facilement la distinction entre la demande du formulaire et l'envoi de la réponse. 

\textbf{TP 1}

Dans le fichier que l'on nommera \texttt{form.html}, créer un formulaire HTML utilisant la méthode POST, ayant aucune action et  demandant à l'utilisateur d'entrer son nom.

\par 

Note : \par 
La balise form aura donc : 
\begin{itemize}
  \item Un attribut \texttt{method} dont la valeur est \texttt{POST}
  \item Un attribut \texttt{action} dont la valeur est \texttt{""}
\end{itemize}

Exemple : Dans le fichier \texttt{form.html}

\begin{minted}[bgcolor=bghtml]{html}
<html>
  <head>
    <meta charset="utf-8">
    <title>Bonjour</title> 
  </head>
  <body>
    
<h1>Un splendide formulaire</h1>

<form method="POST" action="">
  <input type=text name="nom">
  <input type="submit" value="Valider"
</form>

  </body>
</html>
\end{minted}


On a un splendide formulaire, maintenant il faut le faire traiter par le serveur.

\par \vspace{5mm}

Pour differencier les differentes méthodes on utilise la methode request, il faut donc verifier qu'elle soit bien importée au départ.


\begin{minted}[bgcolor=bg]{python}
from flask import Flask, request
\end{minted}

Ensuite comme d'habitude, on affecte notre url à la fonction souhaitées.

\begin{minted}[bgcolor=bg]{python}
@app.route('/formulaire/')
def formulaire():
  """ici on traite le formulaire"""
  return render_template("form.html")
\end{minted}


Attention, on doit prevenir notre serveur que sur cette url, il peut recevoir aussi des requetes "POST", on ajoute donc à la route la methode POST


\begin{minted}[bgcolor=bg]{python}
@app.route('/formulaire/', methods=["GET", "POST"])
def formulaire():
  """ici on traite le formulaire"""
  return render_template("form.html")
\end{minted}


Il ne reste plus qu'à récupérer les valeurs passé par le formulaire : 

\begin{minted}[bgcolor=bg]{python}
@app.route('/formulaire/', methods=["GET", "POST"])
def formulaire():
  """ici on traite le formulaire"""
  if request.method == 'POST':
      nom = request.form.get("reponse")
      print(nom) # On affiche le résultat dans la console
      return nom # On affiche aussi le résultat dans une page WEB
  return render_template("form.html")
\end{minted}

0269 61 92 05



\end{document}